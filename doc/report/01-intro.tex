\section{Introduction}
%	Contextualização
Physical annealing is a process through which a melted metal is allowed to cool slowly, thus forming a defect-free crystal with a regular structure. At higher temperatures the atoms move more freely and rearrange their structure easily. As the temperature decreases, the ability for the atoms to move is diminished since they possess less energy. This allows the material to reach a state of minimum energy -- the crystalline form.

Simulated annealing is an iterative method based on the process of physical annealing for solving combinatorial optimization problems. The system starts at a given (high) temperature, which allows the algorithm to choose and follow a worse solution according to a probability function based on the temperature. In each iteration, a cooling function decreases the system temperature, also decreasing the probability for a worse solution to be followed. Without simulated annealing, the algorithm never accepts a worse solution, which causes it to be trapped in local solutions. With simulated annealing, the algorithm is allowed to change the system state from a local solution to a worse state, which eventually leads to a state closer to the global solution.

%	Motivação
This document aims to study the impact of the simulated annealing method in an usual use case (the room assignment problem) and how the solutions obtained react to the variation of the parameters.

%	Objectivos
In this project two versions are implemented to solve this problem: the first being the sequential implementation and the second being a parallel implementation using MPI. Both versions implement two approaches, one using a strictly minimizing method and another using the simulated annealing method. For each version, both approaches are tested using the same matrix of incompatibility values for a range of different numbers of students to be assigned. The simulated annealing approach is also tested using different initial temperature values.